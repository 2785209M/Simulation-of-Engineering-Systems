\documentclass{article}
\usepackage{blindtext}
\usepackage[a4paper, margin=0.2in]{geometry}
\usepackage{graphicx}
\usepackage{amsmath}
\usepackage{float}
\usepackage{caption}
\usepackage{placeins}
\graphicspath{{../MATLAB/}{images/}} % add any directories that may contain the image

\author{James Middleton (2785209M)}
\title{Simulations Report - Robotic Arm System}
\date{November 2025}

\begin{document}
\maketitle

\section*{Introduction} %==========================================================================

\begin{flushleft}
    This report details the derivation for the control systems of a robotic arm. 
    This includes deriving state space models for the Actuator, Gears, and forearm mechanics.
    The actuator $\theta_M$ is used to provide rotation to the forearm deflection angle $\theta_F$.
    This angle of roation is compared to our reference value $\theta_{ref}$ to control the angle of the forearm.
    To represent these equations I used MATLAB and Simulink. This step in the development process is crucial
    to prevent time and resource wasting errors during real world development.
\end{flushleft}

\section{State Space Equations} %==================================================================

\begin{flushleft}
    To represent the dynamics of the robotic arm the state space model was used on equations provided by the lab sheet.
\end{flushleft}

\flushleft{\textbf{Actuator Equations\\}}
\begin{flushleft}
    The first equation represents the rate of change of the current flowing through the actuator ($\frac{di}{dt}$). This drives the 
    angular velocity of the actuator ($\frac{d\theta_M}{dt}$). $V_A$ represents the input voltage to the actuator, 
    $K_E$ is the back emf constant, and R is the resistance of the Actuator.\\
\end{flushleft}
\begin{align}
    L \frac{di}{dt} + Ri + K_E \frac{d\theta_M}{dt} = V_A
    \Rightarrow \frac{di}{dt} = \frac{V_A}{L} -\frac{Ri}{L} - \frac{K_E}{L}\frac{d\theta_M}{dt}
\end{align}

\begin{flushleft}
    The angular acceleration of the Actuator ($\frac{d^2\theta_M}{dt^2}$) is represented by equation 2. This shows the relationship
    between the actuator and the gears where $B_SM$ is the damping coefficient, $\Delta\omega$ is the difference in
    speed between the motor and gear and $K_T$ is the torque constant.
\end{flushleft}
\begin{equation}
    J_M \frac{d^2\theta_M}{dt^2} + B_{SM} (\frac{d\theta_M}{dt} - \frac{d\theta_{G1}}{dt}) = K_T \Rightarrow 
    \frac{d^2\theta_M}{dt^2} = \frac{K_Ti}{J_M} - \frac{B_{SM}}{J_M}(\frac{d\theta_M}{dt} - \frac{d\theta{G1}}{dt})
\end{equation}

\flushleft{\textbf{Gear 1 Mechanics\\}}
\begin{flushleft}
    The angular acceleration of Gear 1 is represented by equation 3. $J_{G1}$ represents the moment of inertia on Gear 1,
    and $B_{SM}\Delta\theta$ again represents the damping coefficient multiplied by the difference in speed between the Actuator and Gears.
\end{flushleft}
\begin{align}
    J_{G1}\frac{d^2\theta{G1}}{dt^2} - B_{SM}\Delta\omega = 0 \Rightarrow
    J_{G1}\frac{d^2\theta{G1}}{dt^2} - B_{SM}(\frac{d\theta_M}{dt} - \frac{d\theta_{G1}}{dt}) = 0 \Rightarrow
    \frac{d^2\theta{G1}}{dt^2} = \frac{B_{SM}}{J_{G1}}(\frac{d\theta_M}{dt} - \frac{d\theta_{G1}}{dt})
\end{align}

\flushleft{\textbf{Forearm Mechanics\\}}
\begin{flushleft}
    The fourth equation represents the mechanics within the forearm. It shows the relationship
    between the angular acceleration of the forearm ($\frac{d^2\theta_F}{dt^2}$), The torque constant on the forearm
    $T_F$ (calculated using the Gear Ratio ($GR$) multiplied by the back emf constant $K_F$),
    the moment of inertia on the forearm ($J_F$), the damping coefficient ($B_{SF}$), the mass and length of the 
    forearm ($m_Fl_F$), the gravitational constant ($g$), and the total rotation of the arm ($\theta_U + \theta_F$).
\end{flushleft}
\begin{align}
    J_F\frac{d^2\theta_F}{dt^2} + B_SF\frac{d\theta_f}{dt} + \frac{m_Fl_F}{2}gsin(\theta_U + \theta_F) = T_F \nonumber \\ \Rightarrow
    \frac{d^2\theta_F}{dt^2} = \frac{T_F}{J_F} - \frac{B_{SF}}{J_F} \frac{d\theta_F}{dt} - \frac{m_Fl_F}{2J_F}gsin(\theta_U + \theta_F) \nonumber \\ \Rightarrow
    \frac{d^2\theta_F}{dt^2} = \frac{GRK_F\theta_{G1}}{J_F} - \frac{B_{SF}}{J_F} \frac{d\theta_F}{dt} - \frac{m_Fl_F}{2J_F}gsin(\theta_U + \theta_F)
\end{align}

{\footnotesize
\noindent
\begin{minipage}[t]{0.4\linewidth}
\raggedright
\textbf{State Variables}\\[6pt]
\[
\begin{aligned}
    x_1 &= i &&\Rightarrow \text{Current Through the motor}\\
    x_2 &= \theta_M &&\Rightarrow \text{Actuator Deflection Angle}\\
    x_3 &= \dot{\theta}_M &&\Rightarrow \text{Actuator Angular Velocity}\\
    x_4 &= \theta_{G1} &&\Rightarrow \text{Angular Deflection of Gear 1}\\
    x_5 &= \dot{\theta}_{G1} &&\Rightarrow \text{Angular Velocity of Gear 1}\\
    x_6 &= \theta_F &&\Rightarrow \text{Angular Deflection of the Forearm}\\
    x_7 &= \dot{\theta}_F &&\Rightarrow \text{Angular Velocity of the Forearm}\\
\end{aligned}
\]
\end{minipage}\hfill
\begin{minipage}[t]{0.2\linewidth}
\center
\textbf{Differentiation}\\[6pt]
\[
\begin{aligned}
    \dot{x}_1 &= \frac{di}{dt}\\
    \dot{x}_2 &= \dot{\theta}_M\\
    \dot{x}_3 &= \ddot{\theta}_M\\
    \dot{x}_4 &= \dot{\theta}_{G1}\\
    \dot{x}_5 &= \ddot{\theta}_{G1}\\
    \dot{x}_6 &= \dot{\theta}_F\\
    \dot{x}_7 &= \ddot{\theta}_F\\
\end{aligned}
\]
\end{minipage}\hfill
\begin{minipage}[t]{0.33\linewidth}
\raggedright
\textbf{Substitution}\\[6pt]
\[
\begin{aligned}
    \dot{x}_1 &= \frac{di}{dt}\\
    \dot{x}_2 &= x_3\\
    \dot{x}_3 &= \frac{K_T\,i}{J_M} - \frac{B_{SM}}{J_M}(x_3-x_5)\\
    \dot{x}_4 &= x_5\\
    \dot{x}_5 &= \frac{B_{SM}}{J_{G1}}(x_3-x_5)\\
    \dot{x}_6 &= x_7\\
    \dot{x}_7 &= \frac{GR\,K_F\,x_4}{J_F} - \frac{B_{SF}}{J_F}x_7 - \frac{m_F\,l_F}{2J_F}g\sin(\theta_U+x_6)
\end{aligned}
\]
\end{minipage}
}

\section{MATLAB Scripting} %=============================================================================

\subsection{Initial Conditions\\} %-------------------------------------------------------------------------

\begin{minipage}{\linewidth}
    \begin{flushleft}
        The initial conditions for the state variables are represented in MATLAB using an array.\\
    \begin{center}
        x[] = [0, 0, 0, 0, 0, $\theta_U$, 0]\\
    \end{center}
        Where $\theta_U$ is equal to 7\textdegree as specified in the documentation (converted to radians using deg2rad()).
    \end{flushleft}
    \begin{flushleft}
        The control of the system is represented by the equation: $V_E = G_C\Delta\theta$ \\ Where $G_C$ is the gain of the controller 
        and $\Delta\theta$ is the difference between $\theta_{ref}$ and $\theta_F$ which is measured by the actuator sensor represented
        by $K_S$. The reference Deflection then passes through the Reference Amplifier which is represented by $K_R$: 
        \[\theta_{ref} = K_Rdeg2rad(55)\]
        \[V_E = G_C(\theta_{ref}-K_Sx_2)\]
        $V_E$ is input to the Gear Compensator ($K_G$) to create $V_A$. This drives the gears and forearm deflection.
        \[V_A = V_EK_G\]
    \end{flushleft}
\end{minipage}

\subsection{Step-size Selection, and Integration\\} %-------------------------------------------------------------

\begin{flushleft}
For the MATLAB simulation the values of the variables are equal to the parameters provided in the lab sheet.
The robot arm function handles the differentiation of our state variables.
The code below shows the state space equatinons translated into MATLAB script:\\
\end{flushleft}

\begin{figure}[htbp!]
    \includegraphics[width=0.7\linewidth]{Robotic_Arm_Equations_Function.png}
    \caption{Robotic Arm State Equations Method}
\end{figure}
\FloatBarrier

\begin{flushleft}
The following loop represents the movement of the arm over time. It uses specific step sizes in a reasonable time interval
to mathematically simulate physical motion. A step size of 0.001 was chosen in a time period of 10 seconds to provide a detailed plot of the movement
whilst keeping processing time to a minimum. Upon each iteration the x and xdot arrays are updated with the new values and saved. 
$V_A$ (the input voltage) and $V_E$ the control voltage are each updated on each iteration and used to calculate the new 
intefgrationg values for x. These arrays are then used to plot the graphs in figure 4.
\end{flushleft}

\begin{figure}[htbp!]
\includegraphics[width=0.65\linewidth]{MATLAB_Loop.png}
\caption{Main Loop}
\end{figure}
\FloatBarrier

\begin{flushleft}
    The integration method for this simulation is Runge-Kutta 4.
    This method was chosen for its accuracy, superior approximation and simplicity compared to other integration methods.\\
\end{flushleft}

\begin{figure}[htbp!]
    \includegraphics[width=0.65\linewidth]{rk4int.png}
    \caption{Runge Kutta Method}
\end{figure}
\FloatBarrier

\subsection{MATLAB Simulation Graphs} %-------------------------------------------------------------------------------

\begin{figure}[htbp!]
\center
\includegraphics[width=0.7\linewidth]{State_Variable_Graphs.png}
\caption{State Variable Graphs}
\end{figure}
\FloatBarrier

\begin{minipage}{\linewidth}
    \begin{flushleft}
        The Motor Current $x_1$ is shown to oscillate between  2 and -2 amps and gradually comes to rest at 0 Amps. 
        This is concurrent with its purppose as it should be able to dynamically drive the Actuator either up or down and 
        will stop supplying current once the arm has reached the desired position.\\
        $\theta_M$ and $\theta_{G1}$ both initially oscillate betwewn approximately 1.5 and -0.5 before coming to rest at approximately
        0.88 radians which is roughly equal to 55\textdegree, which is the reference angle. This is concurrent with expectations
        as $\theta_M$ and $\theta_{G1}$ differ from the reference angle by $\Delta\theta$ due to the fact that we are not using 
        an integral component within the controller.\\
        $\Theta_F$ begins at roughly 0.15rad  and gradually increases to 0.88rad, to match $\theta_{ref}$. $\theta_F$ dffers from the 
        other system responses due to its stronger damping effects ($B_SF$) and a higher moment of inertia ($J_F > J_M$) which leads 
        to a smoother and more gradual response.\\
        $\dot{\theta}_M$, $\dot{\theta}_{G1}$, and $\dot{\theta}_F$ all begin with oscillation and come to rest at 0\textdegree.
        $\theta_M$ oscillates with high and sharp peaks whilst $\theta_G1$ oscillates with almost identical but slightly dampened 
        peaks but reaches steady state at approximately the same time.\\
        These results show that our equations are working as expected. The systems gradually fades over time as $\theta_F$
        approaches $\theta_{ref}$. However, we can see oscillations in the movement of the forearm. This is due to imperfect gain
        and lack of fine tuning in the system.
    \end{flushleft}
\end{minipage}

\section{Simulink} %=========================================================================================

\begin{figure}[htbp!]
    \begin{flushleft}
    Figure 5 details the high level view of the Robotic Arm System Block Diagram. It shows $\theta_{ref}$ being
    input to the Reference Amplifier ($K_R$), Added with $\theta_MKS$ to create $\Delta\theta$, and amplified by the Elbow Control 
    ($GC$) and Gear Compensator ($K_G$) subsystems. 
    \end{flushleft}
    \center
    \includegraphics[width=0.75\linewidth]{Robotic_Arm_Block_Diagram.png}
    \caption{Robotic Arm System Diagram}
\end{figure}
\FloatBarrier

\begin{figure}[htbp!]
    \begin{flushleft}
    Figure 6 shows the Actuator, Gear, and Forearm subsystems. $V_A$ is input to the Actuator to run the motor. the actuator subsystem
    produces $\theta_M$ and $\dot{\theta_M}$. $\Delta\omega$ is shared between the actuator and gear system, derived from the
    difference between $\dot{\theta_M}$ and $\dot{\theta_{G1}}$, and represents the differences in angular velocities. Physically,
    Gear 1 is connected to the Actuator via the actuator's drive shaft.
    \end{flushleft}
    \center
    \includegraphics[width=0.75\linewidth]{Actuator_Gears_Forearm_Block_Diagram.png}
    \caption{Actuator, Gears, and Forearm System Diagram}
\end{figure}
\FloatBarrier

\begin{figure}[htbp!]
    \begin{flushleft}
    Figure 7 shows the Actuator Subsystem. This outputs the motor current ($i$), the $\theta_M$, and $\dot{\theta_M}$. This equation
    utilizes basic gain coefficients and the moment of impulse on the motor, as well as $\Delta\omega$ from comparison between the outputs
    $\dot{\theta_M}$ and $\dot{\theta_{G1}}$, as well as the actuator's internal resistance.
    \end{flushleft}
    \center
    \includegraphics[width=0.75\linewidth]{Actuator_Block_Diagram.png}
    \caption{Actuator System Diagram}
\end{figure}
\FloatBarrier

\begin{figure}[htbp!]
    \begin{flushleft}
    Figure 8 outputs the values of $\theta_{G1}$ and $\dot{\theta{G1}}$. It utilizes $\Delta\omega$ from comparison
    with the Actuator output, and uses a simple gain for the moment of impulse.
    \end{flushleft}
    \center
    \includegraphics[width=0.75\linewidth]{Gears_Block_Diagram.png}
    \caption{Gear Mechanics System Diagram}
\end{figure}
\FloatBarrier

\begin{figure}[htbp!]
    \begin{flushleft}
    Figure 9 outputs $\dot{\theta_F}$ and $\theta_F$. The Gear Ratio, Torque Gain, and moment of impulse are represented by simple gains.
    $\theta_U$ is represented using a constant combined with $\theta_F$ and put through a sin function and a gain block representeing
    the physical dynamics of the forearm calculated using weight, length, and gravity.
    \end{flushleft}
    \center
    \includegraphics[width=0.75\linewidth]{Forearm_Block_Diagram.png}
    \caption{Forearm Mechanics System Diagram}
\end{figure}
\FloatBarrier

\begin{figure}[htbp!]
\begin{flushleft}
    Figure 10 details the graphs obtained by plotting the output variables of the Simulink model. This, when compared to the MATLAB model,
    shows that the calculations and simulations agree very closely. Each method uses Runge-Kutta 4 integration with a step size of 
    0.001. Any small discrepencies may be due to floating point precision errors.
\end{flushleft}
\centering
\includegraphics[width=0.8\linewidth,height=0.9\textheight,keepaspectratio]{Simulink_Robotic_Arm_Graphs.png}
\caption{Simulink Graphs}
\end{figure}
\FloatBarrier

\section{Conclusion} %===========================================================================================

\begin{flushleft}
    This mathematical simulation of a Robotic Arm system precisely captures the behaviours of subsystems and state variables.
    The mathematical model agrees with the block diagram model and the control systems settle on provided reference values. 
    However, there is oscillation in the movement of the robotic arm. this may be due to improper calculation of gain parameters
    such as $GC$. This could be solved by refinement of the calculations.
\end{flushleft}

\end{document}
